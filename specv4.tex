\documentclass{article}
\usepackage{natbib}
\usepackage[english]{babel}
\usepackage[table]{xcolor}

\title{Inductive Logic Programming in Haskell}
\author{Marcus Peacock 0919274}
\date{11$^{\textrm{th}}$ October 2012}

\begin{document}
\maketitle

\section*{Problem}

The area of interest for my project is Inductive Logic Programming (ILP). ILP
brings together machine learning and logic programming with the aim of
producing algorithms which can inductively learn relational descriptions. I am 
interested in creating an ILP system in Haskell.

\section*{Objectives}

My initial objective is to create a simple emperical system. This objective can
be broken down in to research on what exactly is required for an ILP system,
finding or creating a suitable representation for logic, implementing the
required methods and finally creating the search and heuristic system.
Empirical systems are much
less complex than interactive systems and systems which use searching
such as depth-first or breadth-first are simpler than systems with use
algorithms such as A*. Therefore, as I expect the majority of the time on my
project to be spent on the logic aspect, I have set my objective to create a
basic empirical system. If this turns out to take less time than I expected I
would look at improving the searching algorithm of the system and attempt to
make it more efficient.

\section*{Methods \& Research}

The two pieces of work contributing to most of my understanding of ILP are N. Lavrac, S. Dzeroski \cite{lavdzer} and S. Muggleton \cite{mugg}.

My project will be split in to two parts. The initial part of the project will
be researching the theory and methods behind an ILP system and deciding how I
will implement them. The second part of the project will be my Haskell
implementation.

My current understanding is that an ILP system is a system which starts with
some background knowledge and a set of positive and negative examples and
attempts to find a hypothesis which, along with the background knowledge, entails
all the positive examples and none of the negative examples. It is also
important to be able to determine if a hypothesis is correct, preferably a
hypothesis is complete and consistent but noise must be accounted for and being
easily readable by humans is a positive.

The problem can be viewed as a search problem, the system is searching through
all the possible hypothesis and searching for one which fits the necessary
criteria. As it is impractical for all the hypothesis to be generated and
checked, heuristics and search algorithms are used.

An ILP system can either be interactive or empirical. An empirical system
starts with all the background knowledge and examples it will use and attempts
to find a hypothesis. An interactive system can start with some or none of the
information it will use, allows for information to be added and can deal with
working hypothesis.

A search of the hypothesis space an be performed bottom-up or top-down.
Generalisation techniques search in a bottom-up manner: starting from training
examples and use generalization operators. On the other hand, specialization
techniques search the hypothesis space top-down, from the most general to
specific concept descriptions using specialization operators. My system will
be empirical and my understanding is that specialization techniques are better
suited for empirical systems and can deal with noise with suitable heuristics.
It will therefore be required that I research and understand specialization
techniques in order to implement them correctly.

An ILP system requires a language of logic to operate with. I am yet to decide
how I will represent the logic of my system, it may be possible to use
pre-existing modules for logic representation or I may decide to write my own
system in Haskell. Research on what is available will be required to make this
decision.

I will test my system using examples that are used with existing ILP systems
and compare my results with them. I will also compare how quickly my system
runs in comparison. If suitable it would be interesting to see how the results
of different types of ILP system compare and how quickly each type runs in
comparison to mine.

As my project is in Haskell, extra time will be required to learn how to
effectively use Haskell for my project. I have some experience in Haskell but
I chose to do my project in Haskell because I am interested in becoming much
more proficient. I would like to explore various possibilities for writing the
methods for my system and attempt to make them efficient whilst recognising
how human readable they are.

\section*{Resources}

I expect to be able to use vim for text and coding work, along with git and
github for backup purposes and ghc for compiling Haskell.

\section*{Milestones \& Timetable}

By the end of October I expect to have done enough research to be able
to make decisions on the representation I will use for my project and following
that, be able to start implementing the methods required. I aim to have the
logic part of the system to be complete before the start of term two, allowing
the rest of the time to be used on implementing suitable search algorithms and
testing.

\begin{center}
    \begin{tabular}{|l|l|}
    \hline
    Date & Milestone \\
    \hline
    \rowcolor[gray]{0.9}
    11th October 2012 & Initial Project Specification \\
    31st October 2012 & Research complete \\
    \rowcolor[gray]{0.9}
    26th November 2012 & Progress Report \\
    8th December 2012 & Logic representation and methods complete \\
    1st February 2013 & Functional system with suitable examples \\
    \rowcolor[gray]{0.9}
    4th February 2013 & Project Presentation \\
    \rowcolor[gray]{0.9}
    25th April 2013 & Final Report \\
    \hline
    \end{tabular}
\end{center}

\bibliographystyle{plain}
\bibliography{bib}
\end{document}
